\documentclass[final]{beamer}
\usepackage[orientation=landscape,size=a0, scale=1.5]{beamerposter}
\usepackage[spanish, mexico]{babel}
\usepackage{tkz-berge}

\geometry{hmargin=2.5cm}

\usepackage[utf8]{inputenc}

\linespread{1.15}
\usetheme{sharelatex}

\setbeamercolor{alerted text}{fg=color1}

\title[Congreso Virtual de la Sociedad Matemática Mexicana
19 al 23 de octubre de 2020] 
{%
Cálculo del tipo de homotopía de complejos simpliciales
}

\author{Rafael Villarroel Flores}
\institute{Universidad Autónoma del Estado de Hidalgo}
\date{\today}

\begin{document}
\begin{frame}[t]
%==============================================================================
\begin{multicols}{3}

\section{Introducción}

\subsection{Complejos simpliciales}

Los complejos simpliciales proporcionan una manera sencilla de
involucrar la topología en diversas construcciones combinatorias.
Sea $G$ la gráfica diamante:
\begin{figure}[htbp]
  \centering
  \begin{tikzpicture}[scale=3]
  \SetVertexNoLabel
  \GraphInit[vstyle=Normal]
  \pgfmathsetmacro{\unidad}{2}
  \pgfmathsetmacro{\eje}{0.5*sqrt(3)*\unidad}
  \Vertex[x=-\eje,y=0]{a0}
  \Vertex[x=\eje,y=0]{b0}
  \grPath[form=2,rotation=90,prefix=x,RA=\unidad]{2}
  \EdgeFromOneToAll{a}{x}{0}{2}
  \EdgeFromOneToAll{b}{x}{0}{2}
  \AssignVertexLabel[Math]{x}{2,3}
  \AssignVertexLabel[Math]{a}{1}
  \AssignVertexLabel[Math]{b}{4}
\end{tikzpicture}
\caption{Gráfica \(G\) \label{diamante}}
\end{figure}

Un \alert{complejo simplicial} \((X, \Delta)\) consta de un conjunto
\(X\) de \alert{vértices} y un conjunto de subconjuntos de \(X\)
(llamados \alert{simplejos}), tal que si \(\sigma\in\Delta\) y
\(\tau\subseteq\sigma\), entonces \(\tau\in\Delta\).

\section{Construcciones}

\subsection{El complejo de vecindades}

Se tiene un complejo simplicial \(\mathcal{N}(G)\) (ver \cite{ref1}),
donde:
\begin{itemize}
\item el conjunto de vértices es \(V(G)\),
\item \(\sigma\subseteq V(G)\) es un simplejo si los vértices de
\(\sigma\) tienen un vecino común.
\end{itemize}

\begin{figure}[htbp]
  \centering
  \begin{tikzpicture}[scale=3,baseline=(current bounding box.center),
  relleno/.style={fill=blue!20},
  vertices/.style={fill=magenta},
  radiovertice/.style={radius=0.5mm}]
  \pgfmathsetmacro{\unidad}{1.3}
  \pgfmathsetmacro{\altura}{sqrt(3)*\unidad}
  \pgfmathsetmacro{\mediaaltura}{0.5*sqrt(3)*\unidad}
  \draw  (-\unidad, 0) -- (\unidad, 0) -- (0, -\altura) -- cycle;
  \draw[relleno]  (-\unidad, 0) -- (0,-\mediaaltura) -- (0, -\altura) -- cycle;
  \draw[relleno]  (\unidad, 0) -- (0,-\mediaaltura) -- (0, -\altura) -- cycle;
  \draw[vertices] (0, -\mediaaltura) node[above] {\(1\)} circle[radiovertice];
  \draw[vertices] (0, -\altura) node[below] {\(4\)} circle[radiovertice];
  \draw[vertices] (-\unidad, 0) node[left] {\(3\)} circle[radiovertice];
  \draw[vertices] (\unidad, 0) node[right] {\(2\)} circle[radiovertice];
\end{tikzpicture}
\caption{Complejo \(\mathcal{N}(G)\) \label{complejo-ng}}
\end{figure}

\subsection{El complejo de independencia}

Se tiene un complejo simplicial \(I(G)\)
donde:
\begin{itemize}
\item el conjunto de vértices es \(V(G)\),
\item \(\sigma\subseteq V(G)\) es simplejo si no existen aristas entre
vértices de \(\sigma\).
\end{itemize}

\begin{figure}[htbp]
\centering
\begin{tikzpicture}
  [scale=3,relleno/.style={fill=blue!20},
  vertices/.style={fill=magenta},
  radiovertice/.style={radius=0.5mm}]
  \pgfmathsetmacro{\unidad}{1.5}
  \pgfmathsetmacro{\altura}{sqrt(3)*\unidad}
  \pgfmathsetmacro{\mediaaltura}{0.5*sqrt(3)*\unidad}
  \draw[vertices] (0, 0) -- (\unidad, 0);
  \draw[vertices] (-0.5*\altura, 0.5*\unidad) node[above] {$2$} circle[radiovertice];
  \draw[vertices] (-0.5*\altura, -0.5*\unidad) node[below] {$3$} circle[radiovertice];
  \draw[vertices] (0, 0) node[below] {\(1\)} circle[radiovertice];
  \draw[vertices] (\unidad, 0) node[below] {\(4\)} circle[radiovertice];
\end{tikzpicture}
\caption{Complejo \(I(G)\) \label{complejo-i}}
\end{figure}


\subsection{El complejo de completas}

Se tiene un complejo simplicial \(\Delta(G)\) donde:
\begin{itemize}
\item el conjunto de vértices es \(V(G)\),
\item \(\sigma\subseteq V(G)\) es simplejo si \(\sigma\) induce una
subgráfica completa de \(G\).
\end{itemize}

\begin{figure}[htbp]
\centering
\begin{tikzpicture}
  [scale=3,relleno/.style={fill=blue!20},
  vertice/.style={fill=magenta},
  radio/.style={radius=0.5mm}]
  \pgfmathsetmacro{\unidad}{1}
  \draw[relleno]  (0, \unidad) -- ({sqrt(3)*\unidad}, 0) -- (0, -\unidad) -- cycle;
  \draw[relleno]  (0, \unidad) -- ({-sqrt(3)*\unidad}, 0) -- (0,
  -\unidad) -- cycle;
  \draw[vertice] (0, \unidad) node[above] {\(2\)} circle[radio];
  \draw[vertice] (0, -\unidad) node[below] {\(3\)} circle[radio];
  \draw[vertice] ({-sqrt(3)*\unidad}, 0) node[left] {\(1\)} circle[radio];
  \draw[vertice] ({sqrt(3)*\unidad}, 0) node[right] {\(4\)} circle[radio];
\end{tikzpicture}
\caption{Complejo \(\Delta(G)\) \label{complejo-g}}
\end{figure}

\subsection{El complejo de grado acotado}

Sean \(G\) una gráfica con vértices
\(V(G)=(v_{1},v_{2}\ldots,v_{n})\) y
\(\lambda=(\lambda_{1},\lambda_{2},\ldots,\lambda_{n})\) una
sucesión de enteros no negativos. Se tiene un complejo simplicial
\(\mathrm{BD}^{\lambda}(G)\) donde:
\begin{itemize}
\item el conjunto de vértices es \(E(G)\),
\item \(H\subseteq E(G)\) es simplejo si el grado de \(v_{i}\) en la
gráfica inducida \(G[H]\) es menor o igual a \(\lambda_{i}\).
\end{itemize}

\begin{figure}[htbp]
\centering
\begin{tikzpicture}[scale=3,baseline=(current bounding box.center),
  relleno/.style={fill=blue!20},
  vertice/.style={fill=magenta},
  radio/.style={radius=0.5mm}]
  \pgfmathsetmacro{\unidad}{1.3}
  \draw[relleno] (0, 0) -- (\unidad, \unidad) -- (-\unidad, \unidad) -- cycle;
  \draw[relleno] (0, 0) -- (\unidad, -\unidad) -- (-\unidad, -\unidad) -- cycle;
  \draw (\unidad, -\unidad) -- (\unidad, \unidad);
  \draw (-\unidad, -\unidad) -- (-\unidad, \unidad);
  \draw[vertice] (0, 0) node[right] {\(23\)} circle[radio];
  \draw[vertice] (\unidad, \unidad) node[above right] {\(13\)} circle[radio];
  \draw[vertice] (\unidad, -\unidad) node[below right] {\(34\)} circle[radio];
  \draw[vertice] (-\unidad, -\unidad) node[below left] {\(12\)} circle[radio];
  \draw[vertice] (-\unidad, \unidad) node[above left] {\(24\)} circle[radio];
\end{tikzpicture}
\caption{Complejo \(\mathrm{BD}^{\lambda}(G)\), con $\lambda=(1,2,2,1)$ \label{complejo-bg}}
\end{figure}

\section{Herramientas para el cálculo}

\subsection{Subespacio contraíble}

\begin{figure}[htbp]
  \centering
  \begin{tikzpicture}[scale=2,baseline=(current bounding box.center),
  relleno/.style={fill=blue!20},
  vertice/.style={fill=magenta},
  radio/.style={radius=0.5mm}]
  \pgfmathsetmacro{\unidad}{1.3}
  \draw[relleno] (0, 0) -- (\unidad, \unidad) -- (-\unidad, \unidad) -- cycle;
  \draw[relleno] (0, 0) -- (\unidad, -\unidad) -- (-\unidad, -\unidad) -- cycle;
  \draw (\unidad, -\unidad) -- (\unidad, \unidad);
  \draw (-\unidad, -\unidad) -- (-\unidad, \unidad);
  \draw[vertice] (0, 0) node[right] {\(23\)} circle[radio];
  \draw[vertice] (\unidad, \unidad) node[above right] {\(13\)} circle[radio];
  \draw[vertice] (\unidad, -\unidad) node[below right] {\(34\)} circle[radio];
  \draw[vertice] (-\unidad, -\unidad) node[below left] {\(12\)} circle[radio];
  \draw[vertice] (-\unidad, \unidad) node[above left] {\(24\)} circle[radio];
\end{tikzpicture}
\begin{tikzpicture}[scale=2,baseline=(current bounding box.center)]
  \draw[thick, ->] (-2,0)  -- (2,0);
  \draw (0,0) node[above] {homotópico a};
  \draw (4,0) circle[radius=1.2cm, thick];
  \draw (6.4,0) circle[radius=1.2cm, thick];
\end{tikzpicture}
\end{figure}

\subsection{Vértices dominados (ver \cite{prisner})}

\subsection{Teoría discreta de Morse (ver \cite{ref3})}

\section{Bibliografía}

\begin{thebibliography}{99}

\bibitem{ref1} Lovsáz (1978):  Kneser's conjecture, chromatic number and homotopy.

\bibitem{prisner} Prisner (1992): Convergence of iterated clique graphs.

\bibitem{ref3} Forman (1998): Morse theory for cell complexes.
  
\bibitem{ref2} Larrión, Pizaña, V., (2013): Iterated clique graphs and bordered compact surfaces.



\end{thebibliography}

\end{multicols}

\end{frame}
\end{document}
